\section{Electric Field Solver}
The accurate calculation of the Lorentz force, which governs the movement of charged particles, requires determining the electric field at each computational node. Generally, the electric field is determined by the gradient of the electric potential (see Eq. \ref{E to Phi}). In the one-dimensional case, this relationship simplifies to the derivative of $\phi$ with respect to $x$, as shown in the following equation:

\begin{equation}
E = -\frac{\partial \phi}{\partial x}
\end{equation}

Using the Taylor series expansion once again for the adjacent nodes $i+1$ (see Eq. \ref{Eq: Taylor i+1}) and  $i-1$ (see Eq. \ref{Eq: Taylor i-1}), the resulting expression can be obtained:

\begin{equation}
    f_{i+1} - f_{i-1} = 2\Delta x \frac{\partial f}{\partial x} + \text{HOT}
\end{equation}

This expression can be rearranged to approximate the first derivative using the standard central-difference formula as:

\begin{equation}\label{Eq: Electric field 1d central}
    \frac{\partial f}{\partial x} \approx \frac{f_{i+1} - f_{i-1}}{2\Delta x}
\end{equation}

This approximation is second-order accurate, but since it involves both $i+1$ and $i-1$ terms, it can only be applied to internal nodes. For Boundary nodes we can use Equation \ref{Eq: Taylor i+1} and Equation \ref{Eq: Taylor i-1} directly discarding the second derivative term. However, those terms are only first order accurate leading to a higher innacurary at the boundary compared to the inner nodes. Alternativly, we can introduce the equivalent Taylor series for the estimate value at $i+2$ or respectively $i-2$:
\begin{align}
    f_{i+2} &= f_i + (2\Delta x) \frac{\partial f}{\partial x} + \frac{(2\Delta x)^2}{2} \frac{\partial^2 f}{\partial x^2} \\
    f_{i-2} &= f_i - (2\Delta x) \frac{\partial f}{\partial x} + \frac{(2\Delta x)^2}{2} \frac{\partial^2 f}{\partial x^2}
\end{align}

A mathematically useful approach is to subtract Equation \ref{Eq: Taylor i+1} four times:
\begin{align}
      f_{i+2} - 4 f_{i+1} &=  f_i + (2\Delta x) \frac{\partial f}{\partial x} + 2(\Delta x)^2\frac{\partial^2 f}{\partial x} - 4 f_i -4 \Delta x\frac{\partial f}{\partial x} - \frac{4 (\Delta x)^2}{2} \frac{\partial^2 f}{\partial x^2}\nonumber\\
      &= -3 f_i - 2 \Delta x \frac{\partial f}{\partial x} \\
      \intertext{This results in a second-order accurate expression for the solution at the starting boundary:}
      \text{Forward:} \quad \quad \frac{\partial f}{\partial x} &\approx \frac{-3f_i + 4f_{i+1} - f_{i+2}}{2\Delta x}\\
      \intertext{Using the same approach for the Taylor expansion at the node $i- 2$ and subtracting Equation \ref{Eq: Taylor i-1} four times leads to the solution at the outermost node:}
      \text{Backward:}\quad \quad \frac{\partial f}{\partial x} &\approx \frac{f_{i-2} - 4f_{i-1} + 3f_i}{2\Delta x}
\end{align}

To make \acs{ES PIC} simulations applicable for practical scenarios, it is necessary to implement the electric field solver in three dimensions. For this purpose, the nabla operator ($\nabla$) is employed to cover all three spatial dimensions, providing a comprehensive expression for the electric field in 3D:
\begin{align}
    \boldvec{E} =  -\left( \frac{\partial \phi}{\partial x} \hat{e}_1 + \frac{\partial \phi}{\partial y} \hat{e}_2 + \frac{\partial \phi}{\partial z} \hat{e}_3 \right)
\end{align}

By incorporating Equation \ref{Eq: Electric field 1d central}, the electric field components can be expressed using the central difference scheme for each spatial direction. Here, the counting variables $i$, $j$, and $k$ correspond to the discretized grid points in the $x$-, $y$-, and $z$-dimension, respectively. This leads to the detailed expression shown below:

\begin{align}
    \boldvec{E} = \left( \frac{\phi_\mathrm{i-1,j,k} - \phi_\mathrm{i+1,j,k}}{2\Delta x} \right) \hat{e}_1 + \left( \frac{\phi_\mathrm{i,j-1,k} - \phi_\mathrm{i,j+1,k}}{2\Delta y} \right) \hat{e}_2 + \left( \frac{\phi_\mathrm{i,j,k-1} - \phi_\mathrm{i,j,k+1}}{2\Delta z} \right) \hat{e}_3
\end{align}

Exemplarily, the boundary nodes in the $x$-dimension with a total of $n_\mathrm{i}$ nodes can be calculated as follows:
\begin{align}
\text{Forward:}&\quad\quad\left( E_\mathrm{x} \right)_\mathrm{i = \mathrm{0}} = \frac{3\phi_\mathrm{\mathrm{0}, j, k} + 4\phi_\mathrm{\mathrm{1}, j, k} - \phi_\mathrm{\mathrm{2}, j, k}}{2\Delta x}\\
\text{Backward:}&\quad\quad    \left( E_\mathrm{x} \right)_\mathrm{i = n_\mathrm{i}-1} = \frac{-\phi_\mathrm{n_\mathrm{i}-3, j, k} + 4\phi_\mathrm{n_\mathrm{i}-2, j, k} - \phi_\mathrm{n_\mathrm{i}-1, j, k}}{2\Delta x}
\end{align}


\begin{comment}

\begin{align}
    f_{i-2} - 4f_{i-1} &= f_i + (2\Delta x)\frac{\partial f}{\partial x} + 2(\Delta x)^2 \frac{\partial^2 f}{\partial x^2} - 4f_i - 4\Delta x \frac{\partial f}{\partial x} - \frac{4(\Delta x)^2}{2} \frac{\partial^2 f}{\partial x^2}\\
    &= -3f_i + 2\Delta x \frac{\partial f}{\partial x}\\
    \intertext{Umstellen der Gleichung führt zu:}
\end{align}


We use again taylor expansion for any function:
\begin{align}
f_{i+1} &= f_i + \frac{\Delta x}{1} \frac{\partial f}{\partial x} + \frac{(\Delta x)^2}{2} \frac{\partial^2 f}{\partial x^2} + \text{HOT}\\
f_{i-1} &= f_i - \frac{\Delta x}{1} \frac{\partial f}{\partial x} + \frac{(\Delta x)^2}{2} \frac{\partial^2 f}{\partial x^2} + \text{HOT}
\end{align}



\begin{align}
    \text{Forward:}  \quad\quad \frac{\partial f}{\partial x} &\approx \frac{f_{i+1} - f_i}{\Delta x}\\
    \text{Backward:}  \quad\quad\frac{\partial f}{\partial x} &\approx \frac{f_i - f_{i-1}}{\Delta x}
\end{align}



Actually dont have to write Equations again down just ref via \ref{Eq: Taylor i+1} for i+1 and \ref{Eq: Taylor i-1} for i-1
we subtract the second equation from the first. This gives us

- Unterscheiden zwischen:\\
1. linker node -> forward\\
2. Rechter noder -> backward\\ 
3. Central node -> normal\\

\end{comment}